\chapter{Computational Model}
\label{sec:ComputationalModel}

This chapter explores the finite state machine computational model, illustrating the system's behavior and introducing a tailored version for drone mission planning known as Task Automata.

\section{Computational Model - Finite State Machine}
We briefly discuss the concept of finite state machines (FSM) and their application in modeling drone behaviors.

\subsection{State}
\begin{itemize}
    \item Indicates the agent's current activity and status.
    \item The agent executes relevant commands in its present state, managing operations from initiation to completion.
\end{itemize}

\subsection{Transition Function}
\begin{itemize}
    \item Manages the agent's state transitions in response to specific events.
    \item These events enable the agent to dynamically adjust its mission activities.
\end{itemize}

\subsection{Trigger Event}
\begin{itemize}
    \item Clearly defined within an FSM's design, each state change is associated with a particular event.
    \item Events can vary widely, from simple signals to complex conditional triggers.
\end{itemize}

The suitability of FSMs for drone missions is evident as they facilitate state transitions (or tasks) based on environmental cues or timed events detected by the drone, effectively managing the mission's workflow.

\section{Computational Model - Task Automata}
We delve into \textbf{Task Automata}, a state machine approach specifically designed for drone missions, breaking down the mission execution process and discussing each component in detail.



\subsection{State} In each mission, the drone agent begins at the \textbf{Start State}, transitions through various \textbf{Task States} via transition functions, and eventually terminates at the \textbf{Terminate State}.
\begin{itemize}
    \item \textbf{Start State:} This is where all missions start. It is a blank state that does not do anything but transitions.
    
    \item \textbf{Task State:}  This state defines an individual task to execute for the drone agent. When reached, it will execute all the specified task commands.
    \begin{itemize}

        \item \textbf{Detect Task:} This task instructs the drone to locate objects along a designated path using specific parameters. Completion occurs once the drone has navigated all specified waypoints.
        \item \textbf{Track Task:} This task directs the drone to continuously follow a specific object class, utilizing a selected model. The task is considered complete if the drone loses track of the target for a duration of 10 seconds.
        \item \textbf{Avoidance Task:} This task involves guiding the drone from one location to another while it avoids obstacles along a pre-set route. The task is completed when the drone reaches all the designated waypoints.

    \end{itemize}
    
    \item \textbf{Terminate State:} This is where all missions end. The drone agent will return to home in this state.
\end{itemize}



\subsection{Transition Function}
This section outlines the types of transition functions in Task Automata that direct state changes during drone operations.

\textbf{Transition Type:}
\begin{itemize}
    \item \textbf{Explicit Transition:} This transition is defined by the user.
    \begin{itemize}
        \item \textbf{Start Transition:} Transitions the drone unconditionally from the start state to a task state. There is only one start transition in Task Automata.
        \item \textbf{Task Transition:} Transitions the drone from one task state to another, triggered by a specified event.
    \end{itemize}

    \item \textbf{Implicit Transition:} If a task transition does not specify a subsequent task state to execute upon completion, the agent will implicitly transition to the terminate state after executing all commands within the current task state and without any further task transitions being triggered. If specified, the transition is considered invalid.
\end{itemize}

\subsection{Trigger Event}
A trigger event specifies under what circumstances the \textbf{Task Transition} will be triggered.

\textbf{Events in Drone Operation:}
\begin{itemize}
    \item \textbf{Timeout:} This event triggers after a specified duration of time has passed.
    \item \textbf{Detection:} This event triggers when a specified class of object is detected.
    \item \textbf{Done:} This event triggers when the task has reached its conclusion.
\end{itemize}


\section{Task Automata Evaluation}
This section evaluates the computational capabilities of Task Automata within the framework of drone mission planning.

\subsection{Computational Limitations and Strengths}
Task Automata, as a specialized form of finite state machine, is tailored for handling drone-specific tasks and scenarios. While it demonstrates significant strengths in managing state transitions based on environmental cues and mission-specific events, it inherently lacks the broader computational capabilities of more general models like Turing Machines. Task Automata does not support general-purpose programming features such as complex mathematical computations, general-type conditional branching, or concurrency. Furthermore, it does not include capabilities similar to malloc() or new() found in more expansive programming languages, nor is it designed to simulate the infinite tape of a Turing Machine. These limitations highlight that while Task Automata is effective within its domain, it is not Turing complete.

\subsection{Contextual Computational Completeness}
Considering the specific needs of drone mission planning, the design of Task Automata focuses on simplicity, efficiency, and effectiveness in a highly specialized context. Within the realm of drone operations, Task Automata may not need to simulate the infinite tape of a Turing Machine, as the scope and requirements of missions are usually well-defined and constrained. The primary goal is not expressiveness but operational efficiency and reliability.


\subsection{Implications for DroneDSL}
The specialized nature of Task Automata makes it a suitable model for DroneDSL, providing enough functionality to efficiently manage drone missions without the overhead of unnecessary complexity. This simplification results in better compile-time error checking and a more intuitive understanding for users who are experts in the domain rather than in programming. Thus, while DroneDSL and Task Automata sacrifice universal computational power, they gain in terms of accessibility and practical utility for specific applications.

\section{Conclusion}
Task Automata serves as an effective computational model for DroneDSL, balancing the needs of specialized drone mission planning with the simplicity required for practical application. Although not Turing complete, its design reflects a deliberate trade-off that prioritizes domain-specific efficacy over general computational flexibility.
