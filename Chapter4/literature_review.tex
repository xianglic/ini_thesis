\chapter{Literature Review}
\label{sec:Literature Review}

\section{State of the Art}

This chapter synthesizes key contributions to domain-specific languages (DSLs) for robotic applications, highlighting significant advancements in DSL design for adaptability, mission planning, and operational autonomy.

\textbf{Lucas Alves et al.} introduces DRESS-ML, a domain-specific language designed to enhance the modeling of drone-based applications by incorporating self-adaptive behaviors and exceptional situations, moving beyond traditional models that only account for expected flight plans and limited environmental conditions. Leveraging the principles of Behavior-driven development (BDD) and Aspect-oriented Programming (AOP), DRESS-ML allows for detailed specification of system behaviors and adaptation strategies to ensure resilience and continuous operation in diverse and unpredictable environments.  \cite{Alves2022DRESSML}.

\textbf{Davide Di Ruscio et al.} introduces a family of domain-specific languages designed to specify missions for multi-robot systems, focusing on creating models that are technology-independent, analyzable, executable, and extensible to new application areas. These languages aim to make robotic systems more accessible to non-technical operators by bringing the language closer to the problem domain, thus facilitating the democratization of robotics\cite{Ruscio}.

\textbf{Adrian Rutle et al.} presented CommonLang, a DSL for universal robot programming across varied hardware. Utilizing model-driven engineering, it converts high-level directives into robot-specific code, boosting programming efficiency and cross-platform interoperability \cite{Rutle}.

\textbf{André C. Santos et al.} introduced a DSL for adaptive mobile robot navigation, segregating adaptation logic from core application code to improve modularity and system maintainability. The language's flexibility was validated through smartphone-assisted navigation tests \cite{santos}.

\textbf{Flavia Cavalcanti et al.} presented "Flail: A Domain Specific Language for Drone Path Generation" proposed a novel approach for defining drone flight paths using an HTC Vive, enabling intuitive 3D trajectory design. This DSL simplifies flight control and was demonstrated through drone simulations and flight tests \cite{Cavalcanti}.

\textbf{Chao Cao et al.} introduced an aerial planning behavior tree, a structured approach to drone behavior management responding to environmental and internal states. This method supports autonomous real-time decision-making, optimizing mission execution and safety \cite{Cao}.
