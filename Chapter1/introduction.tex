\chapter{Introduction}
\label{sec:intro}

This thesis introduces DroneDSL, a domain-specific language designed to enhance the effectiveness and efficiency of drone mission planning, particularly in reconnaissance operations. Current methods in drone mission planning face significant challenges, including the complexity of integrating software and hardware functionalities. These complexities not only complicate operations for pilots but also limit the potential for code reusability among developers. To address these issues, DroneDSL provides an abstraction layer that simplifies the interaction with drone systems and supports developer efficiency through reusable code modules.

Furthermore, existing drone-specific languages are primarily designed for straightforward, linear mission plans and do not support complex logical operations such as looping and conditional execution. DroneDSL addresses this gap by incorporating a finite state machine that enables dynamic mission execution. This allows drones to adapt their operations dynamically in response to changing conditions, significantly enhancing the expressiveness and operational capabilities of drones in reconnaissance missions and beyond.

\section{Contributions of the Thesis}
The primary contributions of this thesis are:
\begin{itemize}
    \item The development of DroneDSL, which abstracts away complex drone control issues and provides a platform for reusing code, thereby increasing developer productivity.
    \item Enhancement of the drone mission planning process through a language that supports complex logical structures like loops and conditionals, allowing for dynamic and adaptive mission execution.
\end{itemize}

\section{Overview}
The structure of this thesis is organized as follows:
\begin{itemize}
    \item \textbf{Chapter 2: Background} - Discusses the relevant concepts of drone technology, including their types, components, and the basic principles of drone operation.
    \item \textbf{Chapter 3: Problem Statement} - Identifies the challenges in current drone mission planning and outlines the research goals.
    \item \textbf{Chapter 4: Literature Review} - Reviews existing work in the area of domain-specific languages for drones and other robotic applications.
    \item \textbf{Chapter 5: Computational Model} - Describes the computational models used, including finite state machines and task automata, which are central to DroneDSL.
    \item \textbf{Chapter 6: DroneDSL} - Provides a detailed explanation of DroneDSL, including its syntax and operational semantics.
    \item \textbf{Chapter 7: Implementation} - Discusses the implementation details of DroneDSL, covering aspects from preprocessing and compilation to runtime execution.
    \item \textbf{Chapter 8: Future Work} - Outlines future directions for extending DroneDSL and enhancing its capabilities.
    \item \textbf{Chapter 9: Conclusion} - Concludes the thesis work.
    \item \textbf{Bibliography} - Lists the references used throughout the thesis.
    \item \textbf{Appendix A} - Includes supplementary materials or detailed information that supports the thesis.
\end{itemize}
